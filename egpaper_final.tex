% we are going to use this template to present our work in Hauptseminar CG  

\documentclass[10pt,twocolumn,letterpaper]{article}

\usepackage{iccv}
\usepackage{times}
\usepackage{epsfig}
\usepackage{graphicx}
\usepackage{amsmath}
\usepackage{amssymb}
\usepackage{indentfirst}

% Include other packages here, before hyperref.

% If you comment hyperref and then uncomment it, you should delete
% egpaper.aux before re-running latex.  (Or just hit 'q' on the first latex
% run, let it finish, and you should be clear).
\usepackage[breaklinks=true,bookmarks=false]{hyperref}

\iccvfinalcopy % *** Uncomment this line for the final submission

\def\iccvPaperID{****} % *** Enter the ICCV Paper ID here
\def\httilde{\mbox{\tt\raisebox{-.5ex}{\symbol{126}}}}

% Pages are numbered in submission mode, and unnumbered in camera-ready
%\ificcvfinal\pagestyle{empty}\fi
\setcounter{page}{1}
\begin{document}

%%%%%%%%% TITLE
\title{An Overview Of Immersive Scientific Visualization}

\author{Zhongyuan Yu\\
{\tt\small algoyu@163.com}
% For a paper whose authors are all at the same institution,
% omit the following lines up until the closing ``}''.
% Additional authors and addresses can be added with ``\and'',
% just like the second author.
% To save space, use either the email address or home page, not both
\and
Stefanie Krell\\
{\tt\small stefanie.krell@mailbox.tu-dresden.de}
}

\maketitle
%\thispagestyle{empty}


%%%%%%%%% BODY TEXT
\section{Introduction}

Scientific Visualization is the Visualization of scientific phenomena. That includes he visualization of flow, particles, terrain, volume and tensors. The purpose is to enable scientists to understand, illustrate and get insight from their data. The data is often very large and most of the time 3D.  The use of stereoscopic images can improve the depth cue and the perception of the spatial relationships which might be crucial for scientist when analyzing data.

%\section{Related Disciplines}

\setlength{\parindent}{1pc}
Virtual reality can give us a sense of presence or immersion in an 3D environment.It is a class of computer-controlled multisensory communication technologies that allow more intuitive interactions with data and involve human senses in new ways. Nowadays head mounted displays
(HMDs) like Oculus Rift1 or HTC Vive2 are widely available. That is one way to experience VR. Another Way are systems like the CAVE. It is an immersive virtual reality environment where projectors are directed to between three and six of the walls of a room-sized cube. It uses motion capture system, which records the real time position of the user,for interaction. Also, CAVE2 was released in October 2012. When using VR for Scientific Visualization the scientists can explore the data in a more direct way which might lead to insights they would otherwise not get. 

%\setlength{\parindent}{1pc}
%\textbf{Rendering}:The process of generating an image from a model, by means of computer programs.It has uses in architecture, video %games, simulators, movie or TV visual effects, and design visualization.Rendering process can computationally expensive.So,many rendering algorithms have been researched to optimize it.
% I would not write about rendering here

%\section{Classical Applications and Equipments}


%CAVE:An immersive virtual reality environment where projectors are directed to between three and six of the walls of a room-sized cube it provides an immersive, stereoscopic environment. It uses motion capture system, which records the real time position of the user,for interaction.Also, CAVE2 was released in October 2012.

%\setlength{\parindent}{1pc}
%Caffeine molecular viewer:A molecular viewer Caffeine supports both standard desktop computers as well as multi-screen IVR systems Support for latest generation of HDMs is currently being developed. Another Molecular Viewer would be VMD.

%\setlength{\parindent}{1pc}
%Curtin HIVE(Hub for Immersive Visualisation and eResearch): An advanced visualisation system to serve the growing demands of researchers and industry for visualisation, virtualisation and simulation capabilities.It is a multi-user, multi-display facility and the screen is illuminated using three 1920*1200 DLP projectors warped and blended.

\section{Difficulties in This Field}
\subsection{Processing Data:}In immersive VR, exploring in a high complexity or a high dimensionality modern data set is difficult. Collaborative investigating in abstract representation of high-dimensional data and feature spaces is even harder. More advanced techniques should be introduced to view feature vectors in a space of tens to hundreds of dimensions.


\setlength{\parindent}{1pc}

\subsection{Rendering:}
General challenges in VR are stereo rendering, the high rendering quality and high frame rates of 90 fps to avoid nausea. Because of these reasons applications reduce precision and density to match the criteria. Thinning respective point clouds or converting them to 3D meshes are a way to go. In [A Point...] performance optimization techniques that speed up the rendering pipeline for 3D point clouds and image optimization techniques that improve image quality are introduced. In [Visualizing large-scale..] they used a distributed, image-parallel algorithm to perform volume rendering of electron density fields and to ray-cast ball-and-stick glyph in one pass to speed up rendering for large-scale Molecular Dynamics simulations.

\setlength{\parindent}{1pc}
 In [Immersive Molec...] they used a 2-phase rendering system that combines omnidirectional stereoscopic ray tracing with high-performance view-dependent rasterization to provide one or multiple users with high-quality immersive visualization. When viewing primary representations of molecular, the performance of the system is significantly degraded with increasing amount of atoms and bonds model. The approach will be infeasible when it increased to the tens of thousands. To solve this, the author of this article [1] implements the system with GPU instance rendering method in Unity3d engine. But it is still not good enough. 

%[] "`A Point-Based and Image-Based Multi-Pass Rendering Technique for Visualizing Massive 3D Point Clouds in VR Environments"'
%[]Reda et al. - 2013 - Visualizing large-scale atomistic simulations in u.pdf
%[Immersive Molec...]Immersive Molecular Visualization with Omnidirectional Stereoscopic Ray Tracing and Remote Rendering
\setlength{\parindent}{1pc}
When dealing with protein secondary structure information, which is available for base structure and for every frame of the molecular model trajectory, the frame count can be very large.

\setlength{\parindent}{1pc}
More advanced rendering methods should be used to solve those problems.

%This paper [1] introduces a solution that the information extraction could be processed on demand instead of being extracted by a parser.

\subsection{Interaction:}Tracing labeled neurons manually is time-consuming, may require months to reconstruct even small portions of the brain. It is more challenging working with image slices with fixed viewpoint. Tools such as Vaa3D and NeuroLucida 360 also have their drawbacks. A better tool with immersive environment for tracing should be build.
\subsection{Displaying:}Current visualization tools are often incompatible with large screen multiuser display formats with stereoscopic 3D applications generally running an order of magnitude slower than on high end desktop PCs. Thus, faster displaying algorithms specialized for large screens should be introduced. 


\end{document}
