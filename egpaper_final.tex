% we are going to use this template to present our work in Hauptseminar CG  

\documentclass[10pt,twocolumn,letterpaper]{article}

\usepackage{iccv}
\usepackage{times}
\usepackage{epsfig}
\usepackage{graphicx}
\usepackage{amsmath}
\usepackage{amssymb}

% Include other packages here, before hyperref.

% If you comment hyperref and then uncomment it, you should delete
% egpaper.aux before re-running latex.  (Or just hit 'q' on the first latex
% run, let it finish, and you should be clear).
\usepackage[breaklinks=true,bookmarks=false]{hyperref}

\iccvfinalcopy % *** Uncomment this line for the final submission

\def\iccvPaperID{****} % *** Enter the ICCV Paper ID here
\def\httilde{\mbox{\tt\raisebox{-.5ex}{\symbol{126}}}}

% Pages are numbered in submission mode, and unnumbered in camera-ready
%\ificcvfinal\pagestyle{empty}\fi
\setcounter{page}{1}
\begin{document}

%%%%%%%%% TITLE
\title{An Overview Of Immersive Scientific Visualiztion}

\author{Zhongyuan Yu\\
{\tt\small algoyu@163.com}
% For a paper whose authors are all at the same institution,
% omit the following lines up until the closing ``}''.
% Additional authors and addresses can be added with ``\and'',
% just like the second author.
% To save space, use either the email address or home page, not both
\and
Stefanie Krell\\
{\tt\small stefanie.krell@mailbox.tu-dresden.de}
}

\maketitle
%\thispagestyle{empty}


%%%%%%%%% BODY TEXT
\section{Introduction}

Scientific Visualization is the Visualization of scientific phenomena. The purpose is to enable scientists to understand,illustrate and get insight from their data. The field deals with high amount of data. The use of stereoscopic images can improve the depth cue and the perception of the spatial relationships which might be crucial for scientist when analysing data.

\section{Classical Applications and Equipments}

\section{Difficulties in This Field}
\subsection{
}


\end{document}
