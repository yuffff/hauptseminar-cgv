\documentclass[10pt,twocolumn,letterpaper]{article}

\usepackage{iccv}
\usepackage{times}
\usepackage{epsfig}
\usepackage{graphicx}
\usepackage{amsmath}
\usepackage{amssymb}
\usepackage{indentfirst}

% Include other packages here, before hyperref.

% If you comment hyperref and then uncomment it, you should delete
% egpaper.aux before re-running latex.  (Or just hit 'q' on the first latex
% run, let it finish, and you should be clear).
\usepackage[breaklinks=true,bookmarks=false]{hyperref}

\iccvfinalcopy % *** Uncomment this line for the final submission

\def\iccvPaperID{****} % *** Enter the ICCV Paper ID here
\def\httilde{\mbox{\tt\raisebox{-.5ex}{\symbol{126}}}}

% Pages are numbered in submission mode, and unnumbered in camera-ready
%\ificcvfinal\pagestyle{empty}\fi
\setcounter{page}{1}
\begin{document}

%%%%%%%%% TITLE
\title{A Review Of Immersive Scientific Visualization}

\author{Zhongyuan Yu\\
{\tt\small algoyu@163.com}
% For a paper whose authors are all at the same institution,
% omit the following lines up until the closing ``}''.
% Additional authors and addresses can be added with ``\and'',
% just like the second author.
% To save space, use either the email address or home page, not both
\and
Stefanie Krell\\
{\tt\small stefanie.krell@mailbox.tu-dresden.de}
}

\maketitle
%\thispagestyle{empty}


%%%%%%%%% BODY TEXT
\section{Introduction}
Scientific Visualization is the visualization of scientific phenomena. That includes the visualization of flow, particles, terrain, volume and tensors. The purpose is to enable scientists to understand, illustrate and get insight from their data. The data is often very large and most of the time in 3D.  The use of stereoscopic images can improve the depth cue and the perception of the spatial relationships which might be crucial for scientist when analyzing data. 
\setlength{\parindent}{1pc}
Virtual reality can give a sense of presence or immersion in 3D environment. It is a class of computer-controlled multisensory communication technologies that allow more intuitive interactions with data and involve human senses in new ways. Nowadays head mounted displays (HMDs) like Oculus Rift1 or HTC Vive2 are widely available.There are also VR systems like the CAVE: an immersive virtual reality environment where projectors are directed to between three and six of the walls of a room-sized cube. It uses motion capture system, which records the real time position of the user, for interaction. When using VR for Scientific Visualization, scientists can explore the data in ways which might lead to insights they would otherwise not get.
\section{Point Cloud Visualization}
\subsection{Overview}
\subsection{Rendering}
\subsection{Interaction}

\section{Tracing Neurons}
\subsection{Overview}
\subsection{Rendering}
\subsection{Interaction}

\section{Molecular Visualization}
\subsection{Overview}
\subsection{Rendering}
\subsection{Interaction}

\section{Visualization of Atomistic Simulations}
\subsection{Overview}
\subsection{Rendering}
\subsection{Interaction}

\section{Conclusion}

{\small
\bibliographystyle{unsrt}
\bibliography{reviewbib}
}

\end{document}
